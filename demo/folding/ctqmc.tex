%        File: ctqmc.tex
%     Created: Thu Aug 07 02:00 PM 2014 K
% Last Change: Thu Aug 07 02:00 PM 2014 K
%
\documentclass[a4paper]{article}
\title{Hybridization Expansion Continuous-Time Quantum Monte Carlo}
\author{Beomjoon Goh}
%%%%%%%%%%     Packages & Options     %%%%%%%%%%
\usepackage{amsmath}
\usepackage{amssymb}
\usepackage{amsthm}
\usepackage{array}
\usepackage{booktabs}
\usepackage{braket}
\usepackage{dcolumn}
\usepackage{esint}
\usepackage{float}
\usepackage{graphicx}
\usepackage{kotex}
\usepackage{mathtools}
\usepackage[version=3]{mhchem}
\usepackage{paralist}
\usepackage{xy}

\usepackage[pdftex]{hyperref}

\DeclareMathOperator{\Tr}{Tr}
\begin{document}
\maketitle

\section{Introduction} % (fold)
\label{sec:Introduction}
% section Introduction (end)
Formal Derivation of Dynamical Mean Field Theory

\section{Formal Derivation of Dynamical Mean Field Theory} % (fold)
\label{sec:Formal Derivation of Dynamical Mean Field Theory}

\subsection{Nontrivial Limit in High Dimensions} % (fold)
\label{sub:Nontrivial Limit in High Dimensions}
The kinetic term for $d$-dimensional hyper-cubic lattice is 
\begin{equation}
    H_0 = \sum_{<i,j>}\sum_{\sigma} t_{ij} c^{\dagger}_{i\sigma}c_{j\sigma}
        = \sum_{\mathbf{k},\sigma} \epsilon_{\mathbf{k}} n_{\mathbf{k}\sigma}
\end{equation}
Considering only nearest neighbor hopping with amplitude $t_{ij} = -t$ and with
lattice spacing $a$, the dispersion $\epsilon_{\mathbf{k}}$ and the
corresponding density of state are given by
\begin{equation}
    \epsilon_{\mathbf{k}} = -2t \sum_{n=1}^{d} \cos(k_n a)
\end{equation}
\begin{equation}
    N_d(\omega) = \sum_{\mathbf{k}} \delta(\hbar\omega - \epsilon_{\mathbf{k}})
\end{equation}

The central limit theorem implies that in the limit $d \to \infty$,
\begin{equation}
    N_d(\omega) \xrightarrow{d \to \infty} 
        \frac{1}{2t\sqrt{\pi d}}
        \exp \left[ -\left( \frac{\omega}{2t\sqrt{\pi d}} \right)^2 \right]
\end{equation}
If $t$ is not scaled properly, this DOS is trivial. To have non trivial limit,
\begin{equation}
    t \to \frac{t'}{\sqrt{d}},\ t' = \text{const.}
\end{equation}

In this scaled model, on could get many simplifications which can be found with
the perturbation theory in terms of $U$. Stating only the results, the self
energy becomes a purely local quantity
\begin{equation}
    \Sigma_{ij,\sigma}(\omega) \xrightarrow{d \to \infty}
        \Sigma_{ii,\sigma}(\omega) \delta_{ij}
\end{equation}
and the Fourier transform is $k$ independent
\begin{equation}
    \Sigma_{\sigma}(\mathbf{k}, \omega) \xrightarrow{d \to \infty}
        \Sigma_{\sigma}(\omega)
\end{equation}
Consequently, the interacting Greens function becomes
\begin{equation}
    G_{\mathbf{k,\sigma}}(\omega) =
        \frac{1}{\omega - \epsilon_{\mathbf{k}} + \mu - \Sigma_{\sigma}(\omega)}
\end{equation}
and the $k$ dependency comes only from the noninteracting particles dispersion

Lastly, for a homogeneous system, the local Greens function is the
momentum-averaged lattice Greens function
\begin{equation}
    G_{ii\sigma}(\omega) = L^{-1}\sum_{\mathbf{k}} G_{\mathbf{k}\sigma}(\omega)
\end{equation}
% subsection Nontrivial Limit in High Dimensions (end)

\subsection{Construction of the DMFT} % (fold)
\label{sub:Construction of the DMFT}
The following derivation will make use of the so-called cavity method. Starting
point is
\begin{equation}
    \mathcal{Z} = \int \prod_{i\sigma} Dc^{*}_{i\sigma}Dc_{i\sigma} \exp [-S]
\end{equation}
\begin{equation}
    S = \int_{0}^{\beta} d\tau
        \left[
            \sum_{i\sigma} c^{*}_{i\sigma}(\tau)
            \left(\frac{\partial}{\partial\tau} - \mu \right)
            c_{i\sigma}(\tau)
           +\sum_{ij\sigma} t_{ij}c^{*}_{i\sigma}(\tau)c_{j\sigma}(\tau)
           +\sum_{i}Uc^{*}_{i\uparrow}(\tau)c_{i\uparrow}(\tau)
            c^{*}_{i\downarrow}(\tau)c_{i\downarrow}(\tau)
        \right]
\end{equation}
Splitting the action in to three parts,
\begin{equation}
    S = S_0 + \Delta S + S^{(0)}
\end{equation}
\begin{eqnarray}
    S_0 &=&
        \int_{0}^{\beta} d\tau
            \sum_{\sigma} c^{*}_{0\sigma}(\tau)
            \left(\frac{\partial}{\partial\tau} - \mu \right)
            c_{0\sigma}(\tau)
           +Uc^{*}_{0\uparrow}(\tau)c_{0\uparrow}(\tau)
            c^{*}_{0\downarrow}(\tau)c_{0\downarrow}(\tau) \\
    \Delta S &=&
        \int_{0}^{\beta} d\tau
            \sum_{i\sigma}
            [t_{i0}c^{*}_{i\sigma}(\tau)c_{0\sigma}(\tau)
             t_{0i}c^{*}_{0\sigma}(\tau)c_{i\sigma}(\tau)] \\
    S^{(0)} &=&
        \int_{0}^{\beta} d\tau
        \left[
            \sum_{i\neq0 \sigma}
                c^{*}_{i\sigma}(\tau)
                \left(\frac{\partial}{\partial\tau} - \mu \right)
                c_{i\sigma}(\tau)
           +\sum_{ij\neq0 \sigma}
                t_{ij}c^{*}_{i\sigma}(\tau)c_{j\sigma}(\tau)
           +\sum_{i\neq0}Uc^{*}_{i\uparrow}(\tau)c_{i\uparrow}(\tau)
            c^{*}_{i\downarrow}(\tau)c_{i\downarrow}(\tau)
        \right] \\
\end{eqnarray}
The partition function now reads
\begin{equation}
    \mathcal{Z} =
        \mathcal{Z}^{(0)}
        \int \prod_{\sigma} Dc^{*}_{0\sigma}Dc_{0\sigma}
            \exp[-S_{0}] \braket{\exp[-\Delta S]}_{(0)}
\end{equation}
where the ensemble average is
\begin{equation}
    \braket{X}_{(0)} = 
        \frac{1}{\mathcal{Z}^{(0)}} \int\prod_{i\neq0 \sigma}
            Dc^{*}_{i\sigma}Dc_{i\sigma} X \exp [-S^{(0)}]
\end{equation}

Performing the cumulant expansion, and scaling the hopping amplitude, only the 
Greens function
\begin{equation}
    G_{jk\sigma}^{(0)} (\tau_1 - \tau_2) =
        -\braket{T_\tau c_{j\sigma}(\tau_1) c_{k\sigma}^*(\tau_2)}_{(0)}
\end{equation}
or disconnected contributions of products of the Greens function remains, and 
applying the linked cluster theorem (collecting connected ones) one gets the
local action
\begin{eqnarray}
    S_{\text{loc}} &=&
        \Bigg[
            \int_{0}^{\beta} d\tau
                \sum_{\sigma} c^{*}_{0\sigma}(\tau)
                \left(\frac{\partial}{\partial\tau} - \mu \right)
                c_{0\sigma}(\tau)
               +Uc^{*}_{0\uparrow}(\tau)c_{0\uparrow}(\tau)
                c^{*}_{0\downarrow}(\tau)c_{0\downarrow}(\tau) \\
         && +\int_{0}^{\beta} d\tau_1 \int_{0}^{\beta} d\tau_2
                \sum_{j,k \neq 0 \sigma}
                t'_{j0} t'_{k0} G_{jk\sigma}^{(0)} (\tau_1 - \tau_2)
                c^{*}_{0\sigma}(\tau_1) c_{0\sigma}(\tau_2)
         \Bigg]
\end{eqnarray}

Now the hybridization function $\Delta$ and the non interacting local Greens
function or the free ``Weiss'' mean field propagator $\mathcal{G}$ is introduced
\begin{equation}
    \Delta_{\sigma} (\tau_1 - \tau_2) =
        - \sum_{i,j\neq0} t'_{i0} t'_{j0} G_{ij\sigma}^{(0)} (\tau_1 - \tau_2)
\end{equation}
\begin{equation}
    \mathcal{G}_{\sigma}^{-1} (\tau_1 - \tau_2) =
        -\left(\frac{\partial}{\partial\tau_1} - \mu\right)\delta_{\tau_1\tau_2}
        -\Delta_{\sigma}(\tau_1 - \tau_2)
\end{equation}
so that
\begin{equation}
    S_{\text{loc}} =
        \int_{0}^{\beta} d\tau_1 \int_{0}^{\beta} d\tau_2 \sum_{\sigma}
            c^{*}_{\sigma}(\tau_1)
            \mathcal{G}_{\sigma}^{-1} (\tau_1 - \tau_2)
            c_{\sigma}(\tau_2)
       +U \int_{0}^{\beta} d\tau
            c^{*}_{\uparrow}(\tau)c_{\uparrow}(\tau)
            c^{*}_{\downarrow}(\tau)c_{\downarrow}(\tau)
\end{equation}
The local Greens function is, as usual,
\begin{equation}
    G_{\sigma}(i\omega_n) =
        -\frac{1}{\mathcal{Z}}
        \int\prod_{\sigma} Dc_{\sigma}^{*}Dc_{\sigma}
            [c_{\sigma}(i\omega_n)c_{\sigma}^{*}(i\omega_n)] 
            \exp[ -S_\text{loc} ]
\end{equation}
where
\begin{equation}
    \mathcal{Z} = 
        \int\prod_{\sigma} Dc_{\sigma}^{*}Dc_{\sigma} \exp[-S_\text{loc}]
\end{equation}
This local Greens function and the ``Weiss'' mean field obey the Dyson equation
\begin{eqnarray}
    [G_{\sigma}(i\omega_n)]^{-1}
        &=& \mathcal{G}_{\sigma}^{-1}(i\omega_n) -\Sigma_\sigma(i\omega_n) \\
        &=& i\omega_n -\mu -\Delta_\sigma(i\omega_n) -\Sigma_\sigma(i\omega_n)
\end{eqnarray}
Lastly, as we have seen, the lattice Greens function and the local Greens
function satisfies
\begin{equation}
    G_{\sigma}(i\omega_n) = \sum_{\mathbf{k}} G_{\mathbf{k}\sigma}(i\omega_n)
\end{equation}
which completes the full circle.


% subsection Construction of the DMFT (end)
% section Formal Derivation of Dynamical Mean Field Theory (end)

\section{Continuous-Time Quantum Monte Carlo Impurity Solver} % (fold)
\label{sec:Continuous-Time Quantum Monte Carlo Impurity Solver}
\subsection{Hybridisation Expansion} % (fold)
\label{sub:Hybridisation Expansion}
Writing the Anderson impurity model,
\begin{eqnarray}
    H &=&   H_{\text{loc}} + H_{\text{bath}} + H_{\text{hyb}} \\
       &&   H_{\text{loc}} =
                Un_{\uparrow}n_{\downarrow}-\mu(n_{\uparrow}+n_{\downarrow}) \\
       &&   H_{\text{bath}} =
                \sum_{\mathbf{k}\sigma} \epsilon_{\mathbf{k}}
                c_{\mathbf{k}\sigma}^{\dagger}c_{\mathbf{k}\sigma}\\
       &&   H_{\text{hyb}} = 
                \sum_{\mathbf{k}\sigma}
                \left( 
                    V_{\mathbf{k}\sigma}^{*}
                    c_{\mathbf{k}\sigma}^{\dagger}d_{\sigma}
                   +\text{h.c}
                \right) \\
\end{eqnarray}
After switching to an interaction representation in which
\begin{equation}
    O(\tau) = 
        e^{ \tau(H_{\text{loc}} + H_{\text{bath}})}
        O
        e^{-\tau(H_{\text{loc}} + H_{\text{bath}})}
\end{equation}
the partition function becomes
\begin{eqnarray}
    \mathcal{Z} 
        &=& \Tr_{d}\Tr_{c} \left[
                e^{-\beta(H_{\text{loc}} + H_{\text{bath}})} T\exp \left(
                    -\int_{0}^{\beta}d\tau H_{\text{hyb}}(\tau)
                    \right)
                \right] \\
        &=& \sum_{n=0}^{\infty} \int_{0}^{\beta} \cdots \int_{0}^{\beta}
                d\tau_{1} \cdots d\tau_{2n} \frac{1}{(2n)!} \Tr_{d}\Tr_{c} \left[
                    e^{-\beta(H_{\text{loc}} + H_{\text{bath}})}
                    T (-H_{\text{hyb}}(\tau_{2n}))
                    \cdots (-H_{\text{hyb}}(\tau_{1}))
                \right] \\
        &=& \sum_{C} W_C \\
\end{eqnarray}
where $C$ is a collection of time points on $[0,\beta]$, $\sum_{C} =
\sum_{n=0}^{\infty} \int_{0}^{\beta} \cdots \int_{0}^{\beta}$, and $W_C$ is the
rest.

To get a non-zero trace, we need to consider even number of perturbations. Since
each hybridisation operator changes the number of electrons on the impurity site
by $+1$ or $-1$. We need to get back to the same number of electrons on the
impurity and that's only possible for the even perturbation order.

One can simplify more by recognising the hybridisation Hamiltonian consists of 
two part:
\begin{equation}
    H_{\text{hyb}} = H_{\text{hyb}}^{d^{\dagger}} + H_{\text{hyb}}^{d}
\end{equation}
As mentioned above, to get a non-zero $\mathcal{Z}$, the number of operations
in each parts must be the same: Among $2n$, $n$ for each. Thus, there are
$\dfrac{2n!}{n!n!}$ possibilities to distribute those operators on $[0,\beta]$.
\begin{eqnarray}
    \mathcal{Z} 
        &=& \sum_{n=0}^{\infty} 
            \left(
                \int_{0}^{\beta}d\tau_{1} \cdots \int_{0}^{\beta}d\tau_{n}
            \right)
            \left(
                \int_{0}^{\beta}d\tau_{1}' \cdots \int_{0}^{\beta}d\tau_{n}'
            \right)
            \frac{1}{(n!)^2} \Tr_{d}\Tr_{c} \Bigg[
                e^{-\beta(H_{\text{loc}} + H_{\text{bath}})} \\
        &&  (-H_{\text{hyb}}^{d}(\tau_{n}))
            (-H_{\text{hyb}}^{d^{\dagger}}(\tau_{n}'))
            \cdots
            (-H_{\text{hyb}}^{d}(\tau_{1}))
            (-H_{\text{hyb}}^{d^{\dagger}}(\tau_{1}'))
            \Bigg] \\
\end{eqnarray}

For this Anderson impurity model, the time evolution part doesn't contain a
spin flip term (doesn't rotate the spin). Therefore we need the same number of
$d^{\dagger}_\sigma$ and $d_\sigma$ for each spin $\sigma$, so that there are
$\dfrac{n!}{n_{\uparrow}!n_{\downarrow}!}$ possibilities to distribute the 
operator. Writing again explicitly,
\begin{eqnarray*}
    \mathcal{Z}
        &=& \sum_{\{n_\sigma\}} \prod_{\sigma}
            \left(
                \int_{0}^{\beta}d\tau_{1}^\sigma
                \cdots
                \int_{\tau_{n_{\sigma}-1}^\sigma}^{\beta}
                    d\tau_{n_\sigma}^\sigma
            \right)
            \left(
                \int_{0}^{\beta}d\tau_{1}'^{\sigma}
                \cdots
                \int_{\tau_{n_{\sigma}-1}'^\sigma}^{\beta}
                    d\tau_{n_\sigma}'^\sigma
            \right) \\
        &&  \Tr_{d}\Tr_{c} \Bigg[ 
            e^{-\beta(H_{\text{loc}} + H_{\text{bath}})} T \prod_{\sigma}
            \sum_{k_{1}^{\sigma} \cdots k_{n_\sigma}^{\sigma}}
            \sum_{k_{1}'^{\sigma} \cdots k_{n\sigma}'^{\sigma}}
            V_{k_{1}^{\sigma}}^{*}V_{k_{1}'^{\sigma}} \cdots
            V_{k_{n_{\sigma}}^{\sigma}}^{*}V_{k_{n_{\sigma}}'^{\sigma}} \\
        &&  c_{k_{n_\sigma}^{\sigma}\sigma}^{\dagger} (\tau_{n_\sigma}^\sigma)
            d_{\sigma} (\tau_{n_\sigma}^\sigma)
            d_{\sigma}^{\dagger} (\tau_{n_\sigma}'^\sigma)
            c_{k_{n_\sigma}'^{\sigma}\sigma} (\tau_{n_\sigma}'^\sigma)
            \cdots
            c_{k_{1}^{\sigma}\sigma}^{\dagger} (\tau_{1}^\sigma)
            d_{\sigma} (\tau_{1}^\sigma)
            d_{\sigma}^{\dagger} (\tau_{1}'^\sigma)
            c_{k_{1}'^{\sigma}\sigma} (\tau_{n_\sigma}'^\sigma)\\
\end{eqnarray*}

Since the time evolution does not mix the impurity and the bath, we can 
calculate $\Tr_d$ and $\Tr_c$ separately. Denoting $A^{\sigma}_{\text{loc}}$ as
a segement representation of impurity operators of spin $\sigma$ on $[0,\beta]$
and $A^{\sigma}_{\text{bath}}$ as a segement representation of bath operators of
spin $\sigma$ on $[0,\beta]$, one can write as follows. For impurity,
\begin{equation}
    \Tr_{d} \left[ e^{-\beta H_{\text{loc}}} A^{\sigma}_{\text{loc}} \right]
\end{equation}
and for bath,
\begin{equation}
    \left(\mathcal{Z_{\text{bath}}} \frac{1}{\mathcal{Z_{\text{bath}}}} \right)
    \Tr_{c} \left[
        e^{-\beta H_{\text{bath}}} \prod_{\sigma}
        \sum_{k_{1}^{\sigma} \cdots k_{n_\sigma}^{\sigma}}
        \sum_{k_{1}'^{\sigma} \cdots k_{n\sigma}'^{\sigma}}
        V_{k_{1}^{\sigma}}^{*}V_{k_{1}'^{\sigma}} \cdots
        V_{k_{n_{\sigma}}^{\sigma}}^{*}V_{k_{n_{\sigma}}'^{\sigma}}
        A^{\sigma}_{\text{bath}}
        \right]
\end{equation}

The bath part can be seen as $\mathcal{Z_{\text{bath}}}$ times some expectation
value of a $n$-point correlation function for a non-interacting problem. And
this expectation value can be solved exactly via the Wick's theorem. We know
the answer is a determinant of a $n \times n$ matrix (spin block diagonalised
of course):
\begin{equation}
    \prod_{\sigma} \det \left(M_{\sigma}^{-1}\right)
    \quad \text{where,} \ 
    \left(M_{\sigma}^{-1}\right)_{ij} = \Delta_{\sigma} (\tau_{i}' - \tau_{j})
\end{equation}
$\Delta$ is the famous hybridisation function. Now for a collection $C$, we have
\begin{equation}
    W_{C} = 
        \Tr_{d} \left[ e^{-\beta H_{\text{loc}}} A^{\sigma}_{\text{loc}} \right]
        \mathcal{Z_{\text{bath}}} \prod_{\sigma}\det\left(M_{\sigma}^{-1}\right)
        (d\tau)^{2n_\sigma}
\end{equation}
In the segment representation, one can calculates the $Tr_{d}[\cdots]$ as
\begin{equation}
    Tr_{d}[\cdots] = s e^{\mu(\sum_{\sigma}l_{\sigma}) - U l_{\text{over}}}
\end{equation}
with a permutation sign from the time ordering $s$.
% subsection Hybridisation Expansion (end)

\subsection{Monte Carlo Sampling} % (fold)
\label{sub:Monte Carlo Sampling}
This Generates all possible segment configurations through local updates. Simple
moves such as insertion and removal of one segment or one anti segment are shown
to be enough to achieve this.
The detailed balance condition is
\begin{equation}
    W_{C} P(C \to C')  = W_{C'} P(C' \to C)
\end{equation}
\begin{equation}
    P(C \to C') = P^{\text{prob}}(C \to C') P^{\text{acc}}(C \to C')
\end{equation}
With metropolise algorithm, one can calculate the probability for accepting the
move as $\min [ R, 1 ]$ where,
\begin{equation}
    R = \frac{P^{\text{prob}}(C \to C')}{P^{\text{prob}}(C \to C')}
      = \frac{P^{\text{acc}}(C' \to C)}{P^{\text{acc}}(C' \to C)}
        \frac{W_{C'}}{W_{C}} 
\end{equation}
% subsection Monte Carlo Sampling (end)

\subsection{Measurement of the Greens Function} % (fold)
\label{sub:Measurement of the Greens Function}
\begin{equation}
    G(\tau,0)
        = \frac{1}{\mathcal{Z}} \sum_{C} W_{C}^{d(\tau)d^{\dagger}(0)}
        = \frac{1}{\mathcal{Z}} \sum_{C} W_{C}^{(0,\tau)}
            \frac{W_{C}^{d(\tau)d^{\dagger}(0)}}{W_{C}^{(0,\tau)}}
\end{equation}
\begin{equation}
    \frac{W_{C}^{d(\tau)d^{\dagger}(0)}}{W_{C}^{(0,\tau)}}
        = \frac{\det M^{-1}}{\det(M^{(0,\tau)})^{-1}}
        = (M)_{ji}
\end{equation}
\begin{equation}
    G(\tau) = 
        \left\langle \sum_{ij}\frac{1}{\beta}
        \hat{\delta}(\tau, \tau_i - \tau'_j) M_{ji}\right\rangle_{MC}
\end{equation}
% subsection Measurement of the Greens Function (end)

\subsection{Kinetic Energy} % (fold)
\label{sub:Kinetic Energy}
\begin{eqnarray}
    \int_{0}^{\beta} d\tau G(\tau) \Delta(\beta-\tau)
        &=& \int_{0}^{\beta} d\tau 
            \left\langle \sum_{ij}\frac{1}{\beta}
            \hat{\delta}(\tau, \tau_i - \tau'_j) M_{ji}\right\rangle
            \Delta(\beta-\tau) \\
        &=& -\left\langle \sum_{ij}\frac{1}{\beta}
            \Delta(\tau'_j - \tau_i) M_{ji}\right\rangle \\
        &=& -\frac{1}{\beta} \braket{n}
\end{eqnarray}
% subsection Kinetic Energy (end)

% section Continuous-Time Quantum Monte Carlo Impurity Solver (end)
\end{document}
